% /* cSpell:disable */

% arara: pdflatex: { options: ['--output-directory=build', '--recorder'], interaction: nonstopmode, shell: true, synctex: true }
% arara: biber: { options: ['--output-directory=build', '--isbn-normalise'] }
% arara: makeglossaries: { options: ['-d./build'] }

\documentclass{mimosis}

\usepackage{metalogo}

%%%%%%%%%%%%%%%%%%%%%%%%%%%%%%%%%%%%%%%%%%%%%%%%%%%%%%%%%%%%%%%%%%%%%%%%
% Some of my favourite personal adjustments
%%%%%%%%%%%%%%%%%%%%%%%%%%%%%%%%%%%%%%%%%%%%%%%%%%%%%%%%%%%%%%%%%%%%%%%%
%
% These are the adjustments that I consider necessary for typesetting
% a nice thesis. However, they are *not* included in the template, as
% I do not want to force you to use them.

% This ensures that I am able to typeset bold font in table while still aligning the numbers
% correctly.
\usepackage{etoolbox}

%%%%%%%%%%%%%%%%%%%%%%%%%%%%%%%%%%%%%%%%%%%%%%%%%%%%%%%%%%%%%%%%%%%%%%%%
% Hyperlinks & bookmarks
%%%%%%%%%%%%%%%%%%%%%%%%%%%%%%%%%%%%%%%%%%%%%%%%%%%%%%%%%%%%%%%%%%%%%%%%

\usepackage[%
  colorlinks = true,
  citecolor  = RoyalBlue,
  linkcolor  = RoyalBlue,
  urlcolor   = RoyalBlue,
  unicode,
  ]{hyperref}

\usepackage{bookmark}

%%%%%%%%%%%%%%%%%%%%%%%%%%%%%%%%%%%%%%%%%%%%%%%%%%%%%%%%%%%%%%%%%%%%%%%%
% Bibliography
%%%%%%%%%%%%%%%%%%%%%%%%%%%%%%%%%%%%%%%%%%%%%%%%%%%%%%%%%%%%%%%%%%%%%%%%
%
% I like the bibliography to be extremely plain, showing only a numeric
% identifier and citing everything in simple brackets. The first names,
% if present, will be initialized. DOIs and URLs will be preserved.

\usepackage[%
  autocite     = plain,
  backend      = biber,
  doi          = true,
  url          = true,
  giveninits   = true,
  hyperref     = true,
  maxbibnames  = 99,
  maxcitenames = 99,
  sortcites    = true,
  style        = numeric,
  ]{biblatex}

\input{bibliography-mimosis}
\addbibresource{Thesis.bib}

%%%%%%%%%%%%%%%%%%%%%%%%%%%%%%%%%%%%%%%%%%%%%%%%%%%%%%%%%%%%%%%%%%%%%%%%
% Fonts
%%%%%%%%%%%%%%%%%%%%%%%%%%%%%%%%%%%%%%%%%%%%%%%%%%%%%%%%%%%%%%%%%%%%%%%%

\ifxetexorluatex
  \usepackage{unicode-math}
  \setmainfont{EB Garamond}
  \setmathfont{Garamond Math}
  \setmonofont[Scale=MatchLowercase]{Source Code Pro}
\else
  \usepackage[lf]{ebgaramond}
  \usepackage[oldstyle,scale=0.7]{sourcecodepro}
  \singlespacing
\fi

% abbreviations
\newacronym[
    description={Principal component analysis},
]{PCA}  {PCA}   {principal component analysis}
\newacronym{SNF}    {SNF}       {Smith normal form}
\newacronym[
    description={Topological data analysis},
]{TDA}  {TDA}   {topological data analysis}


% glossary
\newglossaryentry{LaTeX}{%
  name        = {\LaTeX},
  description = {A document preparation system},
  sort        = {LaTeX},
}

\newglossaryentry{Real_numbers}{%
  name        = {$\real$},
  description = {The set of real numbers},
  sort        = {Real numbers},
}

\makeindex
\makeglossaries

%%%%%%%%%%%%%%%%%%%%%%%%%%%%%%%%%%%%%%%%%%%%%%%%%%%%%%%%%%%%%%%%%%%%%%%%
% Ordinals
%%%%%%%%%%%%%%%%%%%%%%%%%%%%%%%%%%%%%%%%%%%%%%%%%%%%%%%%%%%%%%%%%%%%%%%%

\makeatletter
\@ifundefined{st}{%
  \newcommand{\st}{\textsuperscript{\textup{st}}\xspace}
}{}
\@ifundefined{rd}{%
  \newcommand{\rd}{\textsuperscript{\textup{rd}}\xspace}
}{}
\@ifundefined{nd}{%
  \newcommand{\nd}{\textsuperscript{\textup{nd}}\xspace}
}{}
\makeatother

\renewcommand{\th}{\textsuperscript{\textup{th}}\xspace}

%%%%%%%%%%%%%%%%%%%%%%%%%%%%%%%%%%%%%%%%%%%%%%%%%%%%%%%%%%%%%%%%%%%%%%%%
% Microquestions
%%%%%%%%%%%%%%%%%%%%%%%%%%%%%%%%%%%%%%%%%%%%%%%%%%%%%%%%%%%%%%%%%%%%%%%%
\newcounter{microquestioncounter}
\counterwithin*{microquestioncounter}{subsection}
\newcommand{\q}[1]{
  \stepcounter{microquestioncounter} 
  \begin{quote} 
    \textbf{
      Q\thesubsection-
      \Roman{microquestioncounter}:
    } 
    {#1}? 
  \end{quote} 
}

\newcommand{\qs}[2]{
  \stepcounter{microquestioncounter} 
  \begin{quote} 
    \textbf{
      Q\thesubsection-
      \Roman{microquestioncounter}:
    } 
    {#1}? 
    \par  
    \textbf{sources:} 
    {#2} 
  \end{quote} 
}

%to turn off micro-questions comment the previous two line and uncomment the next two
%\newcommand{\q}[1]{}
%\newcommand{\qs}[2]{}

%%%%%%%%%%%%%%%%%%%%%%%%%%%%%%%%%%%%%%%%%%%%%%%%%%%%%%%%%%%%%%%%%%%%%%%%
% Incipit
%%%%%%%%%%%%%%%%%%%%%%%%%%%%%%%%%%%%%%%%%%%%%%%%%%%%%%%%%%%%%%%%%%%%%%%%

\title{\texttt{latex-template}}
\subtitle{A minimal, modern \LaTeX{} package for typesetting your thesis}

\begin{document}

\frontmatter
  % !TEX root = ../Thesis.tex

\begin{titlepage}
  \vspace*{5cm}
  \makeatletter
  \begin{center}
    \begin{Huge}
      \@title
    \end{Huge}\\[0.1cm]
    %
    \begin{Large}
      \@subtitle
    \end{Large}\\
    %
    \emph{by}\\
    \@author
    %
    \vfill
    A document submitted in partial fulfillment
    of the requirements for the degree of\\
    \emph{Bachelor or Master of Science}\\
    at\\
    \textsc{Carinthian University of Applied Sciences}
  \end{center}
  \makeatother
\end{titlepage}

\newpage
\null
\thispagestyle{empty}
\newpage

  \include{src/Abstract}

  \tableofcontents

\mainmatter

  \part[A good part]{%
    A good part\\
    %
    \vspace{1cm}
    %
    \begin{minipage}[l]{\textwidth}
    %
    \textnormal{%
      \normalsize
      %
      \begin{singlespace*}
        \onehalfspacing
        %
        You can also use parts in order to partition your great work
        into larger `chunks'. This involves some manual adjustments in
        terms of the layout, though.
      \end{singlespace*}
    }
    \end{minipage}
  }

  % !TEX root = ../Thesis.tex

\chapter{Introduction}

\section{State of the Art}

\section{Study Overview}

\section{Research Questions}

\section{Aim of the Research}

\section{Scientific Approach}

\section{Structure of the Thesis}

  \chapter{Theoretical Background}

\section{Requirements and Definitions}

\section{Term 1}

\section{Term 2}

\section{...}

  % !TEX root = ../Thesis.tex

\chapter{State of Research}

\section{Apporach for Selecting and Analyzing Studies}

\section{Study Overview}

\section{Identifying Research Gaps}

Some text shown in \cite{Bringhurst12} might help \cite{Tufte01}.
But there is also another solution shown in \cite{Tufte90}.

Also test if the glossary is working with \gls{PCA}.
Short form is \acrshort{PCA} and the long is \acrlong{PCA}.
  % !TEX root = ../Thesis.tex

\chapter{Reserach Design}

\section{Detailed Questions and Hypothesis}

\section{Selection and Justification of the Scientific Methodology}

\section{Selection of Models, Data, Prototypes...}

\section{Detailed Description of Methods, Prototypes, Data...}

\section{Approach to work with Methods, Prototypes, Data...}

  % !TEX root = ../Thesis.tex

\chapter{Results}

\section{Details}

\section{Discussion}

\section{Reviewing Scientific Methods}

\section{Implications}

\section{Recommendations}

\section{Further Research}

\section{Critical Appreciation of Methodology and Results}

  % !TEX root = ../Thesis.tex

\chapter{Conclusion}

\section{Summary}

\section{Final Statements}

\section{Outlook}


% This ensures that the subsequent sections are being included as root
% items in the bookmark structure of your PDF reader.
\bookmarksetup{startatroot}
\backmatter

  \begingroup
    \let\clearpage\relax
    \glsaddall
    \printglossary[type=\acronymtype]
    \newpage
    \printglossary
  \endgroup

  \printindex
  \printbibliography

\end{document}
